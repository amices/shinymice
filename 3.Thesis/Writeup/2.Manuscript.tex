\documentclass[Royal,times,sageh]{sagej}

\usepackage{moreverb,url,natbib, multirow, tabularx}
\usepackage[colorlinks,bookmarksopen,bookmarksnumbered,citecolor=red,urlcolor=red]{hyperref}





\begin{document}

\title{Missing the Point: Non-Convergence in Iterative Imputation Algorithms}

\runninghead{Oberman}

\author{H. I. Oberman\affilnum{1}}

\affiliation{\affilnum{1}{Department of Methodology and Statistics, Utrecht University, Utrecht, The Netherlands}}

\corrauth{Hanne Oberman, Sjoerd Groenman building, Utrecht Science Park, Utrecht, The Netherlands.}

\email{\href{mailto:h.i.oberman@uu.nl}{\nolinkurl{h.i.oberman@uu.nl}}}

\begin{abstract}
Iterative imputation is a popular tool to accommodate missing data. While it is widely accepted that valid inferences can be obtained with this technique, these inferences all rely on algorithmic convergence. There is no consensus on how to evaluate the convergence properties of the method. This paper provides insight into identifying non-convergence of iterative imputation algorithms. We conclude that in the case of drawing inference from incomplete data, convergence of the iterative imputation algorithm is often a convenience but not a necessity.
\end{abstract}

\keywords{missing data, iterative imputation, non-convergence, mice}

\maketitle

\hypertarget{overview-of-sections}{%
\section{Overview of sections}\label{overview-of-sections}}

Intro

\begin{itemize}
\item
  This is missingness
\item
  This is how we solve missingness
\item
  For that we need an algorithm, but what if it has not converged?
\item
  That is what we solve in this paper
\end{itemize}

Algorithmic convergence

\begin{itemize}
\item
  What is algorithmic convergence?
\item
  When is it converged? Stationarity versus mixing.
\item
  How is convergence currently investigated? And how should it be investigated? Multivariate versus univariate.
\end{itemize}

Methods

\begin{itemize}
\item
  Which methods are there? And how do they relate to mixing and stationarity?
\item
  \(\widehat{R}\)
\item
  AC
\item
  We can use these on any \(\theta\)
\end{itemize}

The Simulation study, Results, Discussion.

\hypertarget{introduction}{%
\section{Introduction}\label{introduction}}

Anyone who analyzes person-data may run into a missing data problem. Missing data is not only ubiquitous across science, but treating it can also be a tedious task. If a dataset contains just one incomplete observation, calculations are not defined, \textbf{hence} statistical models cannot do fitted to the data. To circumvent this, many statistical packages employ list-wise deletion by default (i.e., ignoring incomplete observations). Unfortunately, this \emph{ad hoc} solution may yield invalid results \citep{buur18}. An alternative is to apply imputation. With imputation, we `fill in' the missing values in an incomplete dataset. Subsequently, the model of scientific interest can be fitted to the completed dataset. By repeating this process several times, a distribution of plausible results may be obtained, which reflects the uncertainty in the data due to missingness. This technique is known as `multiple imputation' \citep[MI;][]{rubin76}. MI has proven to be a powerful technique to draw valid inferences from incomplete data under many circumstances \citep{buur18}.

Figure \ref{fig:diagram} provides an overview of the steps involved with MI. The missing part \(y_{\rm mis}\) of an incomplete dataset is imputed \(m\) times. This creates \(m\) sets of imputed data \(\dot y_{\rm imp, \ell}\), where \(\ell = 1, 2, ..., m\). The imputed data is then combined with the observed data \(y_{\rm obs}\) to create \(m\) completed datasets. On each of these datasets the analysis of scientific interest is performed to estimate \(Q\): the quantity of scientific interest (e.g., a regression coefficient). Since \(Q\) is estimated on each completed dataset, \(m\) separate \(\hat{Q}_\ell\)-values are obtained. Finally, the \(m\) \(\hat{Q}_\ell\)-values are combined into a single pooled estimate \(\bar{Q}\). The premise of multiple imputation is that \(\bar{Q}\) is an unbiased and confidence-valid estimate of the true---but unobserved---inference \citep{rubin96}.

\begin{figure}

{\centering \includegraphics[width=\linewidth]{../Figures/DiagramMI} 

}

\caption{Scheme of the main steps in multiple imputation (where $m=3$)---from an incomplete dataset, to $m$ multiply imputed datasets, to $m$ estimated quantities of scientific interest $\hat{Q}$s, to a single pooled estimate $\bar{Q}$.}\label{fig:diagram}
\end{figure}

A popular method to obtain imputations is to use the `Multiple Imputation by Chained Equations' algorithm, shorthand `MICE'\citep{mice}. With MICE, imputed values \(\dot y_{\rm imp, \ell}\) are drawn from the posterior predictive distribution of the missing values \(y_{\rm mis}\). The algorithm is named after the iterative algorithmic procedure by which imputed values are generated: a multivariate distribution is obtained by iterating over a sequence of univariate imputations. The iterative nature of algorithms like MICE introduces a potential threat to the validity of the imputations: What if the algorithm has not converged? Are the imputations then to be trusted? And can we rely on the inference obtained on the completed data?

These remain open questions, since the convergence properties of iterative imputation algorithms have not been systematically studied \citep{buur18}. There is no scientific consensus on how to evaluate convergence of imputation algorithms \citep{taka17}. \textbf{Moreover, the convergence of chained equations algorithms such as MICE under certain default imputation models (e.g., `predictive mean matching') is an entirely open question \citep{murr18}.} Therefore, algorithmic convergence should be monitored carefully.

Currently, the recommended practice for evaluating the convergence of iterative imputation algorithms is through visual inspection. After running the imputation algorithm for a certain number of iterations, its convergence is monitiored by plotting a scalar summary of the state space of the algorithm \(\theta\) against the iteration number. Typically, the \(\theta\)s under evaluation are chain means and variances---the average and the variance of the imputed values per imputation. Monitoring convergence by inspecting traceplots of these \(\theta\)s may be undesirable for several reasons on two counts: 1) it may be challenging to the untrained eye, and 2) only severely pathological cases of non-convergence may be diagnosed \citep[\(\S\) 6.5.2]{buur18} and, 3) there is not an objective measure that quantifies convergence. Therefore, a quantitative, diagnostic evaluation of convergence would be preferred---although not straightforward. Iterative imputation algorithms such as MICE are special cases of Markov chain Monte Carlo (MCMC) methods. In MCMC methods, convergence is not from a scalar to a point, but from one distribution to another. The values generated by the algorithm (e.g., imputed values) will vary even after convergence \citep{gelm13}. Since MCMC algorithms do not reach a unique point at which convergence is established, diagnostics may only identify signs of \emph{non}-convergence \citep{hoff09}. Several of such non-convergence diagnostics exist, but it is not known whether these are appropriate for iterative imputation algorithms.

In this paper, we study different methods for assessing non-convergence in iterative imputation algorithms. We define several diagnostics and evaluate how these diagnostics could be appropriate for iterative imputation applications. We then adress the impact of inducing non-convergence in iterative imputation algorithms by means of model-based simulation in \texttt{R} \citep{R}. The aim of the simulation is to determine whether unbiased, confidence-valid inferences may be obtained under different levels of non-convergence (\textbf{and at which point convergence may safely be assumed}). Additionally, we will assess how well different non-convergence diagnostics perform in identifying signs of non-convergence (\textbf{and on which scalar summary of the state space of the algorithm these diagnostics should be applied}). We translate the results of the study into guidelines for applied researchers. \textbf{{[}add more societal relevance: what is the consequence of non-convergence? see example of non-convergence in Figure 2 (e.g., biased estimates, invalid inference, etc.){]}} For reasons of brevity, we only focus on the iterative imputation algorithm implemented in the popular \texttt{mice} package \citep{mice} in \texttt{R} \citep{R}.

\textbf{{[}Include sub-questions and make structure of the paper clear!{]}} How can non-convergence be identified diagnostically? Are common MCMC non-convergence diagnostics appropriate for MICE? And if so, which threshold should be used to diagnose non-convergence? How many iterations are sufficient/needed to be able to diagnose non-convergence? Are the default number of iterations sufficient (i.e., 5 in mice, 10 in SPSS and Stata, 30 in mi)? How severe is it when the algorithm has not converged? And what are guidelines for practice? Can the parameter of interest \(Q\) be correct when the algorithm is not (yet) converged, and vice versa?

\hypertarget{some-notation}{%
\subsection{Some notation}\label{some-notation}}

Let \(y\) denote an \(n \times k\) matrix containing the data values on \(k\) variables for all \(n\) units in a sample. The data value of unit \(i\) (\(i = 1, 2, \dots, n\)) on variable \(j\) (\(j = 1, 2, \dots, k\)) may be either observed or missing. The number of units \(i\) in dataset \(y\) with at least one missing value, divided by the total number of units \(n\), is called the missingness proportion \(p_{\rm mis}\). The collection of observed data values in \(y\) is denoted by \(y_{\rm obs}\); the missing part of \(y\) is referred to as \(y_{\rm mis}\). For each datapoint in \(y_{\rm mis}\), we sample \(m \times T\) plausible values, where \(m\) is the number of imputations (\(\ell = 1, 2, \dots, m\)) and \(T\) is the number of iterations (\(t = 1, 2, \dots, T\)) in the imputation algorithm. The collection of samples between the initial value (at \(t=1\)) and the final imputed value (at \(t=T\)) will be referred to as an `imputation chain'.

\hypertarget{algorithmic-non-convergence}{%
\section{Algorithmic non-convergence}\label{algorithmic-non-convergence}}

There are two requirements for convergence of iterative algorithms: mixing and stationarity \citep{gelm13}. Without mixing, imputation chains do not intermingle nicely, indicating that the distribution of imputed values differs across imputations. Chains may be `stuck' at a local optimum, instead of sampling imputed values from the entire predictive posterior distribution of the missing values. This may cause under-estimation of the variance within chains, which results in spurious, invalid inferences. Without stationarity, there is trending within imputation chains. Trending implies that further iterations would yield a systematically lower or higher set of imputations. Iterative imputation algorithms that have not yet reached stationarity, may thus yield biased estimates.

\textbf{{[}SEGWAY to this section{]}} To illustrate what non-convergence looks like in iterative imputation algorithms, we reproduce an example from van Buuren \citeyearpar[\(\S\) 6.5.2]{buur18}. Figure \ref{fig:non-conv} displays a subset of the example: a variable that we will call \(j\). We see two traceplots of the chain means for \(j\)---i.e., the average of the imputed values for \(j\) in each imputation \(\dot y_{\rm imp, \ell}\), plotted against the iteration number. Each line portrays one imputation. The plot on the left-hand side of the figure shows typical convergence of an iterative imputation algorithm. The right-hand side displays pathological non-convergence, induced by purposefully mis-specifying the imputation model. \textbf{In the typical convergence situation, the imputation chains intermingle nicely and there is little to no trending. In the non-convergence plot, there is a lot of trending and some chains do not intermingle.} Importantly, the chain means at the last iteration (the imputed value per imputation \(\ell\)) are very different between the two plots. The algorithm with the mis-specified model yields imputed values that are on average a factor two larger than those of the typically converged algorithm. That means that non-convergence has a impact on the mean of \(j\) in \(\dot y_{\rm imp, \ell}\). This difference (presumably) translates to bias in the \(m\) completed data estimates \(\hat{\mu}_{j,\ell}\), and consequently also to the pooled estimate \(\bar{\mu}_j\). Non-convergence thus leads to biased estimates. This shows the importance of reaching converged states in iterative imputation algorithms.

\begin{figure}

{\centering \includegraphics[width=.49\linewidth]{2.Manuscript_files/figure-latex/non-conv-1} \includegraphics[width=.49\linewidth]{2.Manuscript_files/figure-latex/non-conv-2} 

}

\caption{Typical convergence versus pathological non-convergence. Please note that this is the same data with a different imputation model, leading to different imputations (e.g., see the units on the y-axis).}\label{fig:non-conv}
\end{figure}

\textbf{{[}MERGE with section below{]}}. Inspecting non-convergence through traceplots of chain means and chain variances may be insufficient because they are univariate summaries of the state space of the algorithm, while MICE is not only concerned with a single column, but the entire multivariate distribution of \(y_{\rm imp}\). Ideally, a multivariate \(\theta\) should be monitored. A suggestion by \citet{buur18} for a multivariate \(\theta\) to monitor is the quantity of scientific interest \(Q\) (e.g., a regression coefficient), which usually is multivariate in nature. Implementing this, however, might be somewhat too technical for empirical researchers. Moreover, this scalar summary is not model-independent, i.e., it only applies to one model of scientific interest. Yet, one of the advantages of MI is that the missing data problem and scientific problem are solved independently. On these grounds, such a \(\theta\) can be considered insufficient too. Van Buuren \citeyearpar[\(\S\) 4.5.2]{buur18} also proposed multivariable evaluation of the MICE algorithm through eigenvalue decomposition, building on the work of \citet{mack03}. Eigenvalues of a variance-covariance matrix are a measure of the data's total covariance. Such a \(\theta\) would be a model-independent, multivariate scalar summary to monitor, but it is not implemented in \texttt{mice}.

\textbf{{[}MERGE with section above{]}} Convergence has historically been inspected visually, by monitoring the imputation algorithm over iterations. This is typically done through traceplots. In a traceplot, the iteration number is plotted against a certain \(\theta\). \(\theta\)s are scalar summaries of the state space of the algorithm at a specific iteration. The default scalar summaries in \texttt{mice} are chain means and chain variances. Additionally, researchers may ``monitor some statistic of scientific interest'' \citep[\(\S\) 6.5.2]{buur18}. As \citet{buur18} describes, researchers can specify their quantity of scientific interest \(Q\) (e.g., a regression coefficient) as scalar summary \(\theta\). Such a user-defined scalar summary, however, may be somewhat advanced for empirical researchers. And moreover, it is not universal to all complete data problems. Focusing on the convergence of outcome parameters may influence the iterative imputation procedure in the sense that the model of evaluation favors the model of interest. The downside to these two approaches is that they either focus on the univariate state space, or primarily track the change over the iterations of a multivariate outcome conform the scientific model of interest. Ideally, one would like to evaluate a model-independent parameter that summarizes the multivariate nature of the data. Therefore, we propose a \(\theta\) that summarizes the multivariate state space of the algorithm, but is independent from the model of scientific interest. We propose \(\lambda_{1, \ell}\) as such a scalar summary. We define \(\lambda_{1, \ell}\) as the first eigenvalue of the variance-covariance matrix of the completed data. Let \(\lambda_{1, \ell} \geq \lambda_{2, \ell} \geq ... \geq \lambda_{j, \ell}\) be the eigenvalues of \(\Sigma_\ell\) in each of the \(m\) completed datasets \{\(y_{\rm obs}, \dot y_{\rm imp, \ell}\)\}. \(\lambda_{1, \ell}\) is measure that summarizes the covariances in the completed datasets. \textbf{The first eigenvalue has the appealing property that is not dependent on the substantive model of interest. Eigenvalue decomposition is inspired by \citet{mack03}.}

\hypertarget{non-convergence-diagnostics}{%
\section{Non-convergence diagnostics}\label{non-convergence-diagnostics}}

Non-convergence diagnostics for MCMC algorithms typically identify problems in either one of the two requirements for convergence: mixing or stationarity. Non-stationarity within chains may be diagnosed with e.g., autocorrelation \citep[\(AC\);][]{scha97, gelm13}, numeric standard error \citep[`MC error';][]{gewe92}, or Raftery and Lewis's \citeyearpar{raft91} procedure to determine the effect of trending on the precision of estimates. A widely used diagnostic to monitor mixing between chains is the potential scale reduction factor \(\widehat{R}\) \citep[`Gelman-Rubin statistic';][]{gelm92}. With a recently proposed adaptation, \(\widehat{R}\) might also serve to diagnose non-stationarity, but this has not yet been thoroughly investigated \citep{veht19}. Therefore, we use \(\widehat{R}\) \textbf{{[}WHICH? Explain that we'll use both{]}} and \(AC\) to evaluate mixing and stationarity separately, as recommended by e.g., \citet{cowl96}.

\textbf{{[}Add: we don't use effective sample size and MC error because they assume independent samples \emph{within} chains. In MI, we only use the final estimate. Also, computing n\_eff assumes infinitely many iterations, which is fine in Bayesian analyses where T is often at least one thousand. But the default in iterative imp is often substantially lower.{]}}

\hypertarget{potential-scale-reduction-factor}{%
\subsection{Potential scale reduction factor}\label{potential-scale-reduction-factor}}

The potential scale reduction factor \(\widehat{R}\) was coined by \citet{gelm92}. An updated version of \(\widehat{R}\) has been proposed by Vehtari et al.~\citeyearpar[p.~5]{veht19}. \textbf{This version is better suited to detect non-convergence in the tails of distributions, by using three transformations on the scalar summary that it is applied to. Namely, rank-normalization, folding, and localization. The Vehtari et al.~version of \(\widehat{R}\) may be suitable for iterative imputation. We therefore use their definition of the diagnostic.} Let \(m\) be the total number of chains, \(T\) the number of iterations per chain (\(T\geq2\)), and \(\theta\) the scalar summary of interest (e.g., chain means). For each chain (\(\ell = 1, 2, \dots, m\)), we estimate the variance of \(\theta\), and average these to obtain within-chain variance \(W\).

\begin{align*}
W&=\frac{1}{m} \sum_{\ell=1}^{m} s_{j}^{2}, \text { where } s_{\ell}^{2}=\frac{1}{T-1} \sum_{t=1}^{T}\left(\theta^{(t \ell)}-\bar{\theta}^{(\cdot \ell)}\right)^{2}. 
\end{align*}

\noindent We then estimate between-chain variance \(B\) as the variance of the collection of average \(\theta\) per chain. \textbf{{[}Note that this is not the usual B in MI{]}}

\begin{align*}
B&=\frac{T}{m-1} \sum_{\ell=1}^{m}\left(\bar{\theta}^{(\cdot \ell)}-\bar{\theta}^{(\cdot \cdot)}\right)^{2}, \text { where } \bar{\theta}^{(\cdot \ell)}=\frac{1}{T} \sum_{t=1}^{T} \theta^{(t \ell)} \text{, } \bar{\theta}^{(\cdot \cdot)}=\frac{1}{m} \sum_{\ell=1}^{m} \bar{\theta}^{(\cdot \ell)}. 
\end{align*}

\noindent From the between- and within-chain variances we compute a weighted average, \(\widehat{\operatorname{var}}^{+}\), which approximates the total variance of \(\theta\). \(\widehat{R}\) is then obtained as a ratio between the total variance and the within-chain variance:

\begin{equation*}
\widehat{R}=\sqrt{\frac{\widehat{\operatorname{var}}^{+}(\theta | y)}{W}},
\text{ where } \widehat{\operatorname{var}}^{+}(\theta | y)=\frac{N-1}{N} W+\frac{1}{N} B.
\end{equation*}

We can interpret \(\widehat{R}\) as potential scale reduction factor since it indicates by how much the variance of \(\theta\) could be shrunken down if an infinite number of iterations per chain would be run \citep{gelm92}. This interpretation assumes that chains are `over-dispersed' at \(t=1\), and reach convergence as \(T \to \infty\). Over-dispersion implies that the initial values of the chains are `far away' from the target distribution and each other. When the sampled values in each chain are independent of the chain's initial value, the mixing component of convergence is satisfied. The variance between chains is then equivalent to the variance within chains, and \(\widehat{R}\)-values will be close to one. High \(\widehat{R}\)-values thus indicate non-convergence.

\hypertarget{autocorrelation}{%
\subsection{Autocorrelation}\label{autocorrelation}}

Following the same notation, we define autocorrelation as the correlation between two subsequent \(\theta\)-values within the same chain \citep[p.~147]{lync07}. In this study, we only consider \(AC\) at lag 1, i.e., the correlation between the \(t^{th}\) and \(t+1^{th}\) iteration of the same chain.

\begin{equation*}
AC = \left( \frac{T}{T-1} \right) \frac{\sum_{t=1}^{T-1}(\theta_t - \bar{\theta}^{(\cdot m)})(\theta_{t+1} - \bar{\theta}^{(\cdot m)})}{\sum_{t=1}^{T}(\theta_t - \bar{\theta}^{(\cdot m)})^2}.
\end{equation*}

We can interpret \(AC\)-values as a measure of stationarity. \(AC\)-values close to zero indicate independence between subsequent \(\theta\)s in imputation chains. Negative \(AC\)-values show no threat to the stationarity component of convergence. On the contrary even---negative \(AC\)-values indicate that \(\theta\)-values of subsequent iterations diverge from one another, which may increase the variance of \(\theta\) and speed up convergence. As convergence diagnostic, the interest is therefore in positive \(AC\)-values. Positive \(AC\)-values occur when \(\theta\)-values of subsequent iterations are similar. When high \(\theta\)-values are followed by high \(\theta\)-values, and low \(\theta\)-values are followed by low \(\theta\)-values, we speak of recurrence. Recurrence within imputation chains indicates trending. Ergo, non-stationarity may be diagnosed when \(AC\) values are larger than zero.

\hypertarget{thresholds}{%
\subsection{Thresholds}\label{thresholds}}

It is unlikely that an MCMC algorithm will achieve the ideal values of \(\widehat{R}=1\) and \(AC=0\). Because of its convergence to a distribution, the algorithm will show some signs of non-mixing and non-stationarity even in the most converged state. Upon approximate convergence, the imputation chains intermingle such that the only difference between the chains is caused by the randomness induced by the algorithm (\(\widehat{R} > 1\)), and there is some dependency between subsequent iterations of imputation chains (\(AC > 0\)). In practice, non-convergence is therefore diagnosed when non-convergence diagnostics exeed a certain threshold. The conventional threshold to diagnose non-mixing is \(\widehat{R} > 1.2\) \citep{gelm92}. Another generally accepted threshold is \(\widehat{R} > 1.1\) \citep{gelm13}. And more recently, \citet{veht19} proposed an even more stringent threshold of \(\widehat{R} > 1.01\). The magnitude of \(AC\)-values may be evaluated statistically, using a Wald test with \(AC=0\) as null hypothesis \citep{box15}. \(AC\)-values that are significantly higher than zero indicate non-stationarity. \textbf{{[}think about one-sided test{]}}

\hypertarget{in-practice}{%
\subsection{In practice}\label{in-practice}}

Before we evaluate the performance of the non-convergence diagnostics \(\widehat{R}\) and \(AC\) quantitatively through simulation, we apply them to the example of pathological non-convergence by \citet{buur18} that we reproduced above. We want to be able to draw the same conclusions as we did from visually inspecting the two algorithms. But \(\widehat{R}\) and \(AC\) should at least be able to distinguish between the two algorithms plotted in Figure \ref{fig:non-conv}. From visual inspection, we know that the typically converged algorithm initially portrayed some signs of non-mixing (around \(t=2\)), but intermingled nicely overall. Additionally, there was very little trending. The algorithm with pathological non-convergence showed severe non-mixing, although this gradually improved (from \(t=7\)). Moreover, there was a lot of trending as well initially (up-to \(t=6\)), after which the chains reached a somewhat more stationary state. To assess whether \(\widehat{R}\) and \(AC\) may be appropriate non-convergence identifiers for iterative imputation algorithms, they should identify the same trends as the visual inspection. But at least, the following condition must hold: the methods should indicate worse performance for the mis-specified model with pathological non-convergence than the typically converged algorithm (i.e., higher \(\widehat{R}\)- and \(AC\)-values).

In Figure \ref{fig:diagnostics}A, the chain means from Figure \ref{fig:non-conv} are plotted again---now together. Panel B shows two versions of calculating \(AC\)s applied to the \(\theta\)s of panel A: the default calculation with \texttt{R} function \texttt{stats::acf()} \citep{R}, and manual calculation as the correlation between \(\theta\) in iteration \(t\) and \(\theta\) in iteration \(t+1\). Panel C shows the traditional computation of \(\widehat{R}\), and panel D shows \(\widehat{R}\) as computed by implementing \citet{veht19} 's recommendations. \textbf{{[}Add labels for the two types of Rhat to the figure!{]}}

\begin{figure}

{\centering \includegraphics{2.Manuscript_files/figure-latex/diagnostics-1} 

}

\caption{Convergence diagnostics applied on the imputation algorithms of Figure 2. $\theta$ = chain mean in $ \dot y_{\rm imp, \ell} $.}\label{fig:diagnostics}
\end{figure}

When we look at panel B, we conclude something weird. The \(AC\)-values calculated with the default function indicate equal performance for the typical convergence and the pathological non-convergence (up-to \(t=5\)), while there is obvious trending in the \(\theta\)s of the latter. Moreover, the best convergence (as indicated by the lowest \(AC\)-value) is observed at \(t=2\), but looking at the chain means in panel A, there should be some signs of trending up-to iteration number seven. After consulting the documentation on \texttt{stats::acf()} \citep{R}, we conclude that this \(AC\) function is not suited for iterative imputation algorithms. The function is optimized for performance when \(t\geq50\) \citep{box15}, while the default number of iterations in iterative imputation is often much lower. Therefore, we compute \(AC\) manually, see the solid line. These \(AC\) values indicate non-stationarity as expected. We therefore calculate \(AC\) manually.

\textbf{Describe panel C here and why it is not optimal: NB. In panel C: De grotere variantie overall zorgt er kennelijk voor dat \(\widehat{R}\) groeit. De \(\widehat{R}\) geeft de foutieve boodschap dat convergentie steeds slechter gaat. Verklaring: de overdispersie assumptie klopt niet als er trend is.{]}} The \(\widehat{R}\)-values in panel D do meet the requirements specified above. \(\widehat{R}\) as computed conform \citet{veht19} does indeed indicate less signs of non-convergence as the number of iterations goes up. \textbf{\citep[explain the dip in Rhat values at t=2. Namely, because we can only use 2 of the 3 tricks by][ if the number of iterations is very low (t\textless4). That's why the \(\widehat{R}\)s are more similar to the traditional GR.]{veht19}}

\textbf{In conclusion, we will use \(\widehat{R}\) conform \citet{veht19} and \(AC\) computed manually as non-convergence diagnostics in the simulation study. The diagnostics will be applied to four \(\theta\)s of interest: chain means, chain variances, a quantity of scientific interest, and the first eigenvalue of the variance-covariance matrix.}

\hypertarget{simulation-study}{%
\section{Simulation study}\label{simulation-study}}

The aim of the simulation study is to evaluate the \textbf{impact of inducing non-convergence {[}``early stopping'' en ``proportion of complete cases''{]}} in the MICE algorithm on several quantities of scientific interest \(Q\). Subsequently we will evaluate how well \(\widehat{R}\) and \(AC\) perform in identifying the effects of non-convergence \textbf{{[}and performance of different thetas{]}}. And finally, we will formulate an informed advice on the requirements to safely conclude sufficient convergence in practice. That is, we conclude that convergence is sufficient when each estimate \(\bar{Q}\) is an unbiased and confidence-valid estimate of the corresponding estimand \(Q\). \textbf{{[}and formulate advice on thetas?{]}}

Non-convergence will be induced by: 1) increasing the missingness proportion \(p_{\rm mis}\) in dataset \(y\) incrementally, and 2) terminating the imputation algorithm at a varying number of iterations \(T\). The first set of simulation conditions---the missingness proportions---is chosen to reflect the difficulty of the missingness problem. The underlying assumption is that low missingness proportions lead to quick algorithmic convergence, since there is a lot of information in the observed data. Higher missingness proportions should yield slower convergence. Unless, however, the fraction of missing information is so high that the random component in the imputation algorithm outweighs the information in the observed data. Then, convergence to a stable but highly variable state may be reached instantaneously. We expect that the incremental missingness proportions in our study will result in a corresponding increase in signs of non-convergence.

The assumption inherent to the second set of simulation conditions---the number of iterations---is that terminating the imputation algorithm too early causes non-convergence. Generally, the algorithm will not reach convergence if \(T=1\), because the imputed values in the first iteration (at \(t=1\)) depend on the initial values of the algorithm (which are sampled randomly from the set of observed datapoints). As the number of iterations increases, the imputation chains should become independent of the initial values, until the point at which adding an extra iteration does not lead to a more converged state. We assume that we can induce non-convergence at least in conditions where \(T\) is smaller than the default number of iterations in \texttt{mice}, \(T=5\).

\textbf{{[}SEGWAY to this section{]} \(Q\)s are the targets that will be estimated in each simulation repetition, after first \emph{amputing} the complete data with varying missingness proportions, then imputing the missingness with a varying number of iterations, then estimating the \(Q\)s on the \(m\) completed datasets, and finally pooling the \(m\) estimates \(\hat{Q}_\ell\) to obtain a single \(\bar{Q}\) for each \(Q\). We try to predict the effect of non-convergence (i.e., the difference between \(Q\) and \(\bar{Q}\)) with the convergence diagnostics \(\widehat{R}\) and \(AC\), by applying these methods to several scalar summaries of the state space of the algorithm, \(\theta\)s.}

\hypertarget{hypotheses}{%
\subsection{Hypotheses}\label{hypotheses}}

\begin{enumerate}
\def\labelenumi{\arabic{enumi}.}
\item
  We expect that simulation conditions with a high missingness proportion \(p_{\rm mis}\) and a low number of iterations \(T\) will more often result in biased, invalid estimates of the quantities of scientific interest, \(Q\)s.
\item
  We hypothesize that \(\widehat{R}\) and \(AC\) will correctly identify signs of non-convergence in those simulation conditions where the \(\bar{Q}\)s are \emph{not} unbiased and confidence-valid estimates of the \(Q\)s.
\item
  We hypothesize that the recommended thresholds to diagnose non-convergence with \(\widehat{R}\) (\(\widehat{R} > 1.2\), \(\widehat{R} > 1.1\), and \(\widehat{R} > 1.01\)) may be too stringent for iterative imputation applications. In an empirical study, where \(\widehat{R}\) was used to inform the required imputation chain length, it took as many as 50 iterations to overcome the conventional non-convergence threshold \(\widehat{R}>1.2\). Yet, scientific estimates were insensitive to continued iteration from \(t=5\) onward \citep{lace07}. We therefore suspect that \(\widehat{R}\) may over-estimate signs of non-convergence in iterative imputation algorithms, compared to the validity of estimates. In contrast to this, it may occur that signs of non-convergence are under-estimated by \(\widehat{R}\), in exceptional cases where the initial values of the algorithm are not appropriately over-dispersed \citep[p.~437]{broo98}. In e.g.~\texttt{mice}, initial values are chosen randomly from the observed data, hence we cannot be certain of over-dispersion in the initial values. In practice, we do not expect this to cause problems for identifying non-convergence with \(\widehat{R}\).
\item
  We expect that high \(AC\) values are implausible in iterative imputation algorithms with typical convergence. That is, after only a few iterations, the randomness induced by the algorithm should effectively mitigate the risk of dependency within chains.
\item
  We further hypothesize that multivariate \(\theta\)s are better at detecting non-convergence than univariate \(\theta\)s.
\end{enumerate}

\hypertarget{set-up}{%
\subsection{Set-up}\label{set-up}}

We investigate non-convergence of the MICE algorithm through model-based simulation in \texttt{R} \citep[version 3.6.3;][]{R}. The number of simulation repetitions is 1000. The simulation set-up is summarized in the pseudo-code below. The complete \texttt{R} script of the simulation study is available from \href{https://github.com/gerkovink/shinyMice/tree/master/3.Thesis/}{github.com/gerkovink/shinyMice}.

\begin{verbatim}
# pseudo-code of simulation 
1. simulate data 
for (number of simulation runs from 1 to 1000)
 for (missingness proportions 5%, 25%, 50%, 75% and 95%)
  2. create missingness
  for (number of iterations from 1 to 100)
   3. impute missingness
   4. perform analysis of scientific interest
   5. compute non-convergence diagnostics 
   6. pool results across imputations
   7. compute performance measures
 8. combine outcomes of all missingness proportions
9. aggregate outcomes of all simulation runs 
\end{verbatim}

\hypertarget{data-generating-mechanism.}{%
\subsubsection{Data-generating mechanism.}\label{data-generating-mechanism.}}

In this study, sampling variance is not of interest. Therefore, a single complete dataset may serve as comparative truth in all simulation repetitions \citep{vink14}. The data-generating mechanism is a multivariate normal distribution, representing person-data on three predictor variables in a multiple linear regression problem. Let the predictor space be defined as
\[
\begin{aligned}
\begin{pmatrix}X_1\\
X_2\\
X_3
\end{pmatrix} \sim \mathcal{N}
\begin{bmatrix}
\begin{pmatrix}
12\\
3\\
0.5
\end{pmatrix}\!\!,
\begin{pmatrix}
4 & 4 & 1.8 \\
4 & 16 & 4.8\\
1.8 & 4.8 & 9
\end{pmatrix}
\end{bmatrix}\!\!\text{.}\\[2\jot]
\end{aligned}
\]
A finite population of \(N=1000\) is simulated using the \texttt{mvtnorm} package \citep{mvtnorm}. Subsequently, a fourth variable is constructed as outcome variable \(Y\). For each unit \(i = 1, 2,..., N\), let
\[
Y_i = 1 + 2X_{1i} +.5X_{2i} - X_{3i} + \epsilon_i ,
\]
where \(\epsilon \sim \mathcal{N}(0, 100)\). This results in a complete set \(y\), with observed values for all units \(i\) on all variables \(j\) (\(j = Y, X_1, X_2, X_3\)). From the complete set we obtain the true values of the scientific estimands \(Q\).

\hypertarget{scientific-estimands.}{%
\subsubsection{Scientific estimands.}\label{scientific-estimands.}}

We consider four types of \(Q\)s that are often of interest in empirical research. Namely, two descriptive statistics: the mean \(\mu_j\) and standard deviation \(\sigma_j\) of each variable \(j = Y, X_1, X_2, X_3\). We also consider the regression coefficients of the predictors: \(\beta_1\), \(\beta_2\), and \(\beta_3\) in regression equation
\[Y' = \beta_0 + \beta_1 X_1 + \beta_2 X_2 + \beta_3 X_3,\]
where \(Y'\) is the expected value of the outcome. And finally, the proportion of variance explained by the set of predictors: coefficient of determination \(r^2\) \textbf{(note that lower case \(r\) is used to avoid confusion with non-convergence diagnostic \(\widehat{R}\))}. The true values of these scientific estimands are calculated on the complete data.

\hypertarget{methods.}{%
\subsubsection{Methods.}\label{methods.}}

In each simulation repetition, we \emph{ampute} the complete dataset five times to obtain five different incomplete datasets, \(y\)s. The ratio between \(y_{\rm obs}\) and \(y_{\rm mis}\) in each \(y\) depends on the missingness proportion: \(p_{\rm mis} =.05,.25,.5,.75,.95\). We ampute the complete data with the \texttt{mice} package \citep[function \texttt{mice::ampute()};][]{mice}. We consider all possible univariate and multivariate patterns of missingness, conform a `missing completely at random' missingness mechanism \citep{rubin87}. I.e., the probability to be missing is the same for all \(N \times k\) cells in \(y\), conditional on the missingness proportion \(p_{\rm mis}\).

Missing datapoints in each incomplete dataset \(y\) are imputed with the \texttt{mice()} function \citep{mice}. All imputation procedures are performed with Bayesian linear regression imputation, and five imputation chains (\(m=5\)). Because \(m=5\), each run of the \texttt{mice()} function results in five sets of imputations: \(\dot y_{\rm imp, \ell}\), where \(m = 1, 2, ..., 5\). The number of iterations in each run of the algorithm varies between simulation conditions (\(T=1,2,...,100\)).

From the 5 imputations \(\dot y_{\rm imp, \ell}\) we obtain the chain means and chain variances for each variable \(j\). These will serve as scalar summary \(\theta\) to apply non-convergence diagnostics \(\widehat{R}\) and \(AC\) on. We then use \texttt{mice::complete()} to combine the imputed data \(\dot y_{\rm imp, \ell}\) with the observed data \(y_{\rm obs}\), resulting in five completed datasets \{\(y_{\rm obs}, y_{imp, m}\)\}.

On the 5 completed datasets \{\(y_{\rm obs}, y_{imp, m}\)\}, we:

\begin{itemize}
\item
  Estimate \(\hat{Q}_\ell\)s for \(Q= \mu\) and \(Q=\sigma\) with \texttt{base::mean()}, and \texttt{stats::sd()} \citep{R}. We pool the \(\hat{Q}_\ell\)s with \texttt{base::mean()} to get \(\bar{Q}\)s.
\item
  Compute \(\lambda_{1, \ell}\), to use as \(\theta\) in the two convergence diagnostics. By definition, the first eigenvalue of the variance-covariance matrix, \(\lambda_{1, \ell}\), is equal to the variance of the principal component solution of each \{\(y_{\rm obs}, y_{imp, m}\)\}, which we calculate with \texttt{stats::princomp()} \citep{R}.
\item
  Perform multiple linear regression with \texttt{stats::lm()} \citep{R} to get the \(\hat{Q}_\ell\)s for \(Q=\beta\) and \(Q=r^2\). We pool the \(\hat{Q}_\ell\)s conform \citet{vink14} to get \(\bar{Q}\)s for the \(\beta\)s and \(r^2\) \textbf{{[}check finite pooling of \(r^2\), instead of \texttt{mice::pool.r.squared()}{]}}. Additionally, we use the \(\hat{Q}_\ell\)s of \(Q=\beta\) as \(\theta\) to apply the convergence diagnostics to.
\end{itemize}

We apply non-convergence diagnostics \(\widehat{R}\) and \(AC\) to each scalar summary \(\theta\). We calculate \(\widehat{R}\) by implementing Vehtari et al.'s \citeyearpar{veht19} recommendations, and \(AC\) as the correlation between the \(t^{th}\) and the \((t+1)^{th}\) iteration.

\hypertarget{performance-measures.}{%
\subsubsection{Performance measures.}\label{performance-measures.}}

We assess the performance of the iterative imputation algorithm in each simulation condition by comparing \(\bar{Q}\)s with \(Q\)s. For each \(Q\), we calculate bias as \(\bar{Q} - Q\). For the regression coefficients \(Q=\beta_{1,2,3}\), we also compute the empirical coverage rate (CR). CR is defined as the percentage of simulation repetitions in which the 95\% confidence interval (CI) around \(\bar{Q}\) covers the true estimand \(Q\). Let
\[\text{CI} = \bar{Q} \pm t_{(M-1)} \times SE_{\bar{Q}},\]
where \(t_{(M-1)}\) is the quantile of a \(t\)-distribution with \(m-1\) degrees of freedom, and \(SE_{\bar{Q}}\) is the square root of the pooled variance estimate. If we obtain nominal coverage (CR = 95\%), we can conclude that the \(\bar{Q}\)s are confidence-valid estimates of the \(Q\)s. Finally, we inspect CI width (CIW): the difference between the lower and upper bound of the 95\% confidence interval around \(\bar{Q}\). CIW is of interest because it is a measure of efficiency. Under nominal coverage, short CIs are preferred. Too short CIs, however, indicate under-estimation of the variance in \(\bar{Q}\), \textbf{which may yield inferences with low statistical power if CIW is large}.

With these performance measures, we also evaluate how well the non-convergence diagnostics \(\widehat{R}\) and \(AC\) can identify simulation conditions in which the inferences are affected by non-convergence, i.e., where \(\bar{Q}\) is \emph{not} an unbiased, confidence-valid estimate of \(Q\). \textbf{{[}REPHRASE this:{]} For performance of univariate \(Q\)s, we apply the non-convergence diagnostics to univariate \(\theta\)s. When \(Q\) is multivariate, we look at signs of non-convergence identified in multivariate \(\theta\)s.}

\begin{itemize}
\item
  For \(Q=\mu\), \(\theta\)s are chain means \textbf{{[}ADD that these should be unbiased because MCAR!{]}}.
\item
  For \(Q=\sigma\), \(\theta\)s are chain variances.
\item
  For \(Q=\beta\), \(\theta\)s are estimated \(\beta\)s per imputation.
\item
  For \(Q=r^2\), \(\theta\)s are the first eigenvalue of the variance-covariance matrix, \(\lambda_{1,\ell}\).
\end{itemize}

\hypertarget{results}{%
\section{Results}\label{results}}

\textbf{{[}Add more info about figure legends and axes{]}}

Our results show the impact of \textbf{inducing non-convergence {[}``early stopping'' en ``proportion of complete cases'' {]}} in an iterative imputation algorithm on several scientific estimates, and whether the non-convergence diagnostics \(\widehat{R}\) and \(AC\) identify the signs of non-convergence in the algorithm. For reasons of brevity, we only discuss the non-convergence diagnostics for estimates with the worst performance in terms of bias. Bias in the descriptive statistics (\(Q=\mu\) and \(Q=\sigma\)) is most pronounced for the outcome variable \(Y\). Of the regression coefficients (\(Q=\beta\)) we observe the largest bias for \(\beta_1\) (the effect of \(X_1\) on \(Y\)). There is only one estimate for \(Q=r^2\) to consider. In the figures, we present simulation conditions where the number of iterations is \(1 \leq T\leq50\), since the results are more or less stable for conditions where \(T \geq 30\). Full results are available from \href{https://github.com/gerkovink/shinyMice/tree/master/3.Thesis/1.SimulationStudy}{github.com/gerkovink/shinyMice}.

\hypertarget{quantities-of-scientific-interest}{%
\subsection{Quantities of scientific interest}\label{quantities-of-scientific-interest}}

\begin{figure}

{\centering \includegraphics{2.Manuscript_files/figure-latex/Qs-1} 

}

\caption{Impact of non-convergence on statistical inferences. Depicted are the bias, coverage rate (CR) and confidence interval width (CIW) of the estimates $\bar{Q}$. We present the subset of $Q$s with the worst performance in terms of bias: the mean and standard deviation of $Y$, and the regression coefficient of $X_1$.}\label{fig:Qs}
\end{figure}

Figure \ref{fig:Qs} displays the influence of \textbf{inducing non-convergence {[}``early stopping'' en ``proportion of complete cases''{]}} on the estimated quantities of scientific interest, \(Q\)s. \textbf{{[}Add panel labels, describe panels{]}} Simulation conditions that are impacted by non-convergence portray more extreme bias in the estimated \(Q\)s, and non-nominal coverages.

\hypertarget{qmu.}{%
\subsubsection{\texorpdfstring{\(Q=\mu\).}{Q=\textbackslash mu.}}\label{qmu.}}

The estimated univariate means seem unaffected by the number of iterations in the algorithm. This implies that valid inferences may be obtained with as little as one iteration (\(T \geq 1\)). The magnitude of the bias in \(\bar{Q}\) depends solely on the missingness proportion \(p_{\rm mis}\), with missingness proportions of 75\% and 95\% leading to more extreme biases than other missingness proportions \textbf{{[}note that these are curious{]}}. Unbiased estimates are obtained in all conditions where \(p_{\rm mis}\leq.50\). Note however, that even in the conditions where \(p_{\rm mis}\geq.75\), the bias is small: \(\mu_Y\) is maximally under-estimated by 0.02 units, while the true value of \(\mu_Y\) is 25.81 (\(\sigma_Y = 11.32\)). We conclude that non-convergence does not substantially impact the validity of the inferences for \(Q=\mu\). \textbf{{[}To be expected, because MCAR{]}}

\hypertarget{qsigma.}{%
\subsubsection{\texorpdfstring{\(Q=\sigma\).}{Q=\textbackslash sigma.}}\label{qsigma.}}

The estimated standard deviations are largely affected by the missingness proportions, but not to much by the number of iterations. Valid inferences are obtained in conditions with just two iterations. Interestingly, the bias in first iteration is actually \emph{less} extreme than for any other number of iterations. \textbf{{[}Due to MCAR and randomly drawn initial values?{]}} We may therefore conclude that no iteration at all is needed to obtain approximately unbiased estimates for \(Q=\sigma\). Completely unbiased estimates are only obtained in conditions where \(p_{\rm mis}=.05\). Conditions where \(p_{\rm mis}\geq.25\) are impacted by non-convergence in order of increasing missingness proportion. Similar to the bias for \(Q=\mu\) the magnitude of the bias in \(Q=\sigma\) is negligible. We therefore conclude that neither of the univariate \(Q\)s is affected substantially by non-convergence.

\hypertarget{qr2.}{%
\subsubsection{\texorpdfstring{\(Q=r^2\).}{Q=r\^{}2.}}\label{qr2.}}

Bias in the estimated variance explained depends both on missingness proportions and the number of iterations. The magnitude of the bias clearly follows the incremental order of the missingness proportions. Completely unbiased estimates may only be obtained if the missingness proportion is at most 25\%. Approximately unbiased estimates are observed for any missingness proportion. For \(p_{\rm mis}\leq.25\), the minimum number of iterations is two (\(T\geq2\)), for \(p_{\rm mis}=.50\) it is \(T\geq4\), and \(p_{\rm mis}=.75\) requires \(T\geq5\). The highest number of iterations needed to obtain valid inferences is seven (\(p_{\rm mis}=.95\); \(T\geq7\)). \textbf{{[}interpret sign of bias? ADD what this implies: dat - zelfs bij een gebiasede methode - je maar 7 iteraties nodig hebt om het eindresultaat te bereiken.{]}}

\hypertarget{qbeta.}{%
\subsubsection{\texorpdfstring{\(Q=\beta\).}{Q=\textbackslash beta.}}\label{qbeta.}}

For the estimated regression coefficients, we consider bias, confidence interval width (CIW), and coverage rate (CR) as performance measures. \textbf{{[}Split panels{]}} Unbiased estimates are obtained in conditions where \(p_{\rm mis}\leq.50\) and \(T\geq3\). Approximately confidence-valid estimates are obtained in any condition where \(T\geq3\)---despite of persistant bias in conditions where \(p_{\rm mis}\geq.75\). These approximately nominal coverages are explained by the CIWs: as expected, conditions with higher missingness proportions result in more uncertainty in the estimates (i.e., wider CIs), which diminish the effect of the bias. confidence-valid inferences may be obtained when \(T\geq3\), while we need at least seven iterations to reach valid inferences in terms of bias. Continued iteration after \(T\geq7\) seems futile. \textbf{{[}add max T for other betas?{]}}

\hypertarget{summary}{%
\subsubsection{Summary}\label{summary}}

Overall, we see that conditions with a higher proportion of missingness are affected more severely by non-convergence. Multivariate \(Q\)s are also substantially influenced by the number of iterations. This means that the algorithm has indeed not converged at \(t=1\), and converges gradually as \(t\) moves to \(T\). The point at which an additional iteration does not lead to an improvement in the estimates depends on quantity of scientific interest \(Q\) and the missingness proportion \(p_{\rm mis}\). The highest number of iterations that is necessary to reach this point across all simulation conditions is \(T=7\). This implies that, under the current specifications, no more than seven iterations are necessary to obtain sufficient performance in terms of bias and confidence-validity.

\hypertarget{non-convergence-diagnostics-1}{%
\subsection{Non-convergence diagnostics}\label{non-convergence-diagnostics-1}}

We evaluate the non-convergence diagnostics \(\widehat{R}\) and \(AC\) against the performance measures described above (i.e., bias and confidence-validity of the quantities of scientific interest \(Q\)). The diagnostics should identify those simulation conditions that are negatively impacted by non-convergence. We expect the most signs of non-convergence in conditions where valid estimates are not obtained (high missingness proportions, \(T<7\)), and at least some signs of non-convergence in conditions where estimates are not approximately unbiased (\(p_{\rm mis}=.95\), any number of iterations).

Figure \ref{fig:rhats-acs} displays the non-convergence diagnostics \(\widehat{R}\) and \(AC\) for each of the four \(\theta\)s under consideration. \textbf{{[}Add panel labels and explain how to read the plots{]}}

\begin{figure}

{\centering \includegraphics{2.Manuscript_files/figure-latex/rhats-acs-1} 

}

\caption{Non-convergence diagnostics. [Split up into 2 or 4 figures??]}\label{fig:rhats-acs}
\end{figure}

\hypertarget{theta-chain-mean.}{%
\subsubsection{\texorpdfstring{\(\theta=\) chain mean.}{\textbackslash theta= chain mean.}}\label{theta-chain-mean.}}

\textbf{{[}As expected, \(\widehat{R}\) lowers with the number of iterations. At a cut-point of 1.2 , we have convergence at about 8 iteration, whereas we need 12 and 25 iteration for more conservative cut-off 1.1 and 1.05. We do not reach the very strict 1.01.{]}}

We evaluate the performance of the non-convergence diagnostics applied to chain means against the bias in \(Q=\mu\). Based on the bias in \(Q\), we concluded that one iteration is sufficient to obtain valid inferences, irrespective of the missingness proportion. The non-convergence diagnostics, however, identify some signs of non-convergence in conditions with as many as thirty iterations. This does not correspond to the impact of non-convergence in the univariate \(Q\)s. \textbf{Compared to the multivariate \(Q\)s {[}add before that we evaluate the thetas against all Qs?{]}, \(\widehat{R}\) and \(AC\) applied to the chain means fail to identify the non-convergence in conditions where \(p_{\rm mis}=.95\).}

\hypertarget{theta-chain-variance.}{%
\subsubsection{\texorpdfstring{\(\theta=\) chain variance.}{\textbackslash theta= chain variance.}}\label{theta-chain-variance.}}

The non-convergence diagnostics applied to the chain variances as \(\theta\) yield highly similar results to the chain means. \textbf{{[}Segway to performance of \(Q=\sigma\){]} While the bias in the estimated standard deviations mainly depends on the missingness proportion of each condition, \(\widehat{R}\) and \(AC\) are influenced mostly by the number of iterations. We thus conclude that there is no added value of evaluating non-convergence with chain variances as \(\theta\) in the diagnostics.}

\hypertarget{thetahatq_ell.}{%
\subsubsection{\texorpdfstring{\(\theta=\hat{Q}_\ell\).}{\textbackslash theta=\textbackslash hat\{Q\}\_\textbackslash ell.}}\label{thetahatq_ell.}}

\textbf{{[}We zien weer hetzelfde voor laag-medium, beetje meer voor extreme misisnh data.{]}}

The non-convergence diagnostics are applied to the estimated regression coefficients in each imputation, and evaluated against the bias in the corresponding pooled estimate.

\(\widehat{R}\) indicates signs of non-convergence across all missingness proportions and each number of iterations, in the expected order: simulation conditions with higher missingness proportions and less iterations are identified as the worst converged. The thresholds to diagnose non-convergence are overcome in different conditions, depending on the number of iterations and the missingness proportion. The most lenient threshold (\(\widehat{R}>1.2\)) is overcome at \(T\geq9\) for \(p_{\rm mis} = 0.05\), and \(T\geq12\) for \(p_{\rm mis} = 0.95\). The conventional \(\widehat{R}>1.1\) is surpassed at \(T\geq12\) and \(T\geq24\), respectively.

For the autocorrelations applied to the \(\hat{Q}_\ell\)s, we see an even stronger effect of the missingness proportions. Especially the conditions where \(p_{\rm mis} = 0.75\) and \(p_{\rm mis} = 0.95\) are identified as non-stationary. The threshold used to evaluate whether \(AC=0\), however, is only reached in conditions with at least ten iterations. This suggests that the default number of iterations may be too low to detect non-stationarity with \(AC\). However, the critical value for diagnosing non-convergence with \(AC\) decreases as a function of the number of iterations. The conclusion to identify non-stationarity in conditions where \(T\geq10\) and \(p_{\rm mis} = 0.95\) might therefore be spurious.

Compared to the bias in \(\bar{Q}\), the diagnostics applied to the \(\hat{Q}_\ell\)s over-estimate the severity of the non-convergence.

\hypertarget{summary-1}{%
\subsubsection{Summary}\label{summary-1}}

In general, the diagnostics identify more signs of non-convergence in simulation conditions with a lower number of iterations. The missingness proportions, however, only seem to impact multivariate convergence. When \(\widehat{R}\) and \(AC\) are applied to univariate \(\theta\)s, they identify the same amount of non-convergence for any missingness proportion. When the diagnostics are applied to multivariate \(\theta\)s, they generally identify more signs of non-convergence for higher missingness proportions.

As expected, none of the simulation conditions that we consider resulted in complete convergence of the algorithm (defined as \(\widehat{R}=1\) and \(AC=0\)). However, the thresholds used in practice to diagnose non-convergence seem too conservative for imputation purposes. \(\widehat{R}\) and \(AC\) identify signs of non-convergence in conditions where the \(\bar{Q}\)s are unbiased and confidence-valid estimates of the \(Q\)s. The most recent recommended threshold for diagnosing non-convergence with \(\widehat{R}\) is never overcome in this study (all \(\widehat{R}\)\textgreater1.01). Apparently, this threshold is too stringent for iterative imputation algorithms, and will not be discussed any further.

\hypertarget{thetalambda_1-ell.}{%
\subsubsection{\texorpdfstring{\(\theta=\lambda_{1, \ell}\).}{\textbackslash theta=\textbackslash lambda\_\{1, \textbackslash ell\}.}}\label{thetalambda_1-ell.}}

Finally, we evaluate the non-convergence diagnostics applied to the first eigenvalue of the variance-covariance matrices of the completed data. Similar to the results the the regression coefficients as \(\theta\)s, we see effects of both the missingness proportion and number of iterations in each simulation condition. The most signs of non-convergence are identified in conditions where \(p_{\rm mis}\) is large and \(T\) is small.

Notwithstanding the similarities with \(\theta=\hat{Q}_\ell\), there is a big difference in \(\widehat{R}\)-values for conditions where \(T<4\). In these conditions, \(\widehat{R}\) behaves oppositely of our expectation: conditions with the lowest missingness proportions yield the highest \(\widehat{R}\)-values. In conditions with a higher number of iterations, this artifact disappears. This suggests that the combination of \(\widehat{R}\) and \(\lambda_{1, \ell}\) is only appropriate for imputation algorithms where the number of iterations at least four. Interpreting the \(\widehat{R}\) values of conditions where \(T<3\) may lead to the incorrect conclusion that only conditions with \emph{low} missingness proportions show signs of non-convergence. For applications as a model-independent multivariate \(\theta\), we should therefore only interpret the \(\widehat{R}\)-values when \(T\geq4\). Using this cut-off we conveniently ignore the initial `dip' at \(T=3\) as well. The traditional threshold \(\widehat{R} > 1.2\) is overcome in conditions where \(p_{\rm mis} = .95\) and \(T\geq10\), and for all other missingness proportions in conditions where \(T\geq8\). With the \(\widehat{R} > 1.1\) threshold, the number of iterations are \(T\geq20\) and \(T\geq16\), respectively.

The \(AC\)-values largely depend on the missingness proportion in each condition. Conditions where \(p_{\rm mis}=.95\) show statistically significant autocorrelations for any number of iterations. The other extreme, conditions where \(p_{\rm mis}=.05\), does not yield \(AC\) values above the threshold. The \(AC\) values with this missingness proportion are equivalent for all conditions where \(T\geq6\), which suggests that the algorithm is as stationary as it can be. A similar point is only observed after twenty iterations in conditions with higher missingness proportions.

In comparison to the \(Q\)s, the diagnostics again over-estimate the signs of non-convergence. Conditions with low missingness proportions (e.g., 5\% or 25\% of the complete data) yield unbiased, confidence-valid estimates with as little as two iterations. The convergence diagnostics, however, indicate improved convergence up-to eight iterations according to \(AC\) and thirty to forty iterations according to \(\widehat{R}\). The model-independent \(\theta\) \(\lambda_{1, \ell}\) seems as well-equipped to diagnose signs of non-convergence as \(\theta=\hat{Q}_\ell\). The univariate \(\theta\)s do not differentiate between bias induced by the missingness proportions.

\hypertarget{discussion}{%
\section{Discussion}\label{discussion}}

We arrive back at the original questions that we posed.

\begin{itemize}
\item
  Non-convergence can be identified with non-convergence diagnostics \(\widehat{R}\) and \(AC\). But common thresholds are too conservative. Why? Part of the distribution is already determined by the observed data. in regular MCMC the entire distribution is unknown (\textbf{priors??}). In iterative imputation algorithms we have the same characteristics as MCMC in the limit, but because we combine the known distribution with the unknown distribution, convergence is reached much sooner.
\item
  Inferences are unbiased much quicker than the algorithm has converged. All things univariate are converged almost intantly after initiating the iterative algorithm with a valid imputation model. But less quick with multivariate parameters.
\end{itemize}

In short, iterative imputation algorithms may yield approximately unbiased, confidence-valid estimates under different levels of induced non-convergence. Univariate estimates seem robust against terminating the algorithm early: There is no clear effect of the number of iterations on the bias in estimated means and standard deviations. The bias in these \(Q\)s was only impacted by the missingness proporion \(p_{\rm mis}\), with higher \(p_{\rm mis}\) resulting in more bias. Bias in multivariate estimates depended on both \(p_{\rm mis}\) and the number of iterations. However, the only effect of the number of iterations was observed when the algorithm was terminated at seven iterations or earlier. This suggests that continued iteration beyond seven iterations does not yield better estimates. Yet, the non-convergence diagnostics \(\widehat{R}\) and \(AC\) indicated that the iterative imputation algorithm did not reach a stable state before twenty to thirty iterations. \textbf{{[}MERGE or remove{]} \(\widehat{R}\) and autocorrelation indicate that there are signs of non-convergence in the algorithm up-to at least twenty iterations, while unbiased, confidence-valid estimates may be obtained with as little as one iteration. This observation is in agreement with }one of {[}still add the others!{]}** the simulation hypotheses: \(\widehat{R}\) over-estimates the severity of non-convergence when applied to MI procedures.**

Over-estimation of the signs of non-convergence may be due to the methods (and their thresholds) or due to the \(Q\)s (descriptive and multivariate linear regression, not higher dimensional/more complex \(Q\)s). More complicated \(Q\)s (e.g., higher-order effects or variance components) might show bias, under- or over-coverage at higher \(T\). Which might be why \(\widehat{R}\) and \(AC\) identify non-convergence.

Some other interesting findings are \textbf{{[}REPHRASE{]}}

\begin{itemize}
\item
  Is there more bias in conditions where non-convergence was induced? (YES for multivar., NOPE for univariates).
\item
  Is the point of \emph{sufficient} convergence (at which the \(\bar{Q}\)s are unbiased and confidence-valid estimates of the \(Q\)s) the same point as identified by the non-convergence diagnostics? (NOPE, they are too conservative).
\item
  Is one of the thetas better? (YES, while it doesn't really matter as long as it's multivariate, \(\lambda_{1, \ell}\) is model-independent and therefore preferred).
\item
  What should the thresholds be? (much HIGHER than recommended for \(\widehat{R}\), difficult to say for \(AC\)).
\item
  And the default nr of iterations? (MAYBE increase to 10 to detect significant \(AC\) and be sure not to miss the dip in \(\widehat{R}\)?).
\end{itemize}

\hypertarget{recommendations-for-empirical-researchers}{%
\subsection{Recommendations for empirical researchers}\label{recommendations-for-empirical-researchers}}

After imputation it is always wise to take the following steps:

\begin{enumerate}
\def\labelenumi{\arabic{enumi}.}
\item
  First, check traceplots for signs of pathological non-convergence. Adjust imputation model if necessary.
\item
  Monitor non-convergence diagnostics \(\widehat{R}\) and \(AC\). Do not use the traditional calculation for \(\widehat{R}\), or the \texttt{R} function \texttt{stats::acf()} to compute \(AC\). Instead, calculate \(\widehat{R}\) conform \citet{veht19} and compute autocorrelations manually (see e.g., \href{https://github.com/gerkovink/shinyMice/tree/master/3.Thesis/1.SimulationStudy}{github.com/gerkovink/shinyMice}).
\item
  Track \(\theta\)s over iterations, e.g., the novel \(\theta\) that is `substantive model-independent'.
\item
  Track your own scalar summary of interest (see e.g., \citet{buur18}). Compute \(\widehat{R}\) and autocorrelation values for this scalar summary.
\item
  Use the threshold \(\widehat{R}>1.2\) for mixing, and assess stationarity by plotting the autocorrelation over iterations. Keep iterating until the threshold for mixing is reached, and until a reasonably decreasing asympototic \(AC\) value is obtained. At that point, inferences are unlikely to improve by continued iteration.
\end{enumerate}

\hypertarget{recommendations-for-future-research}{%
\subsection{Recommendations for future research}\label{recommendations-for-future-research}}

Much remains unknown. about non-convergence in iterative imputation algorithms. Further research is needed to investigate the performance of iterative imputation algorithms under clear violation of convergence, e.g.~dependency between predictors (predictors with very high correlations). For example:

\begin{itemize}
\item
  Induce non-convergence with mis-specified imputation model.
\item
  Introduce more difficult missingness problems, i.e., M(N)AR instead of MCAR. That is, proper performance under a `missing completely at random' missingness mechanism is necessary to demonstrate the appropriateness of \(\widehat{R}\) and \(AC\) as non-convergence diagnostics. However, results may not be extrapolated to other missingness mechanisms.
\item
  Imputation technique with questionable convergence properties, e.g., PMM.
\item
  Apply recommendations to empirical data and write a vignette for applied researchers.
\item
  Implement recommendations in imputation software, e.g., evaluation environment for applied researchers.
\item
  Develop an additional convergence diagnostic unique for iterative imputation: \(AC\) across imputations, instead of iterations.
\end{itemize}

In short, we have shown that an iterative imputation algorithm can yield correct outcomes, even when a converged state has not yet formally been reached. Any further iterations would then burn computational resources without improving the statistical inferences. To conclude, in the case of drawing inference from incomplete data, convergence of the iterative imputation algorithm is often a convenience but not a necessity. \textbf{{[}Our study found that - in the cases considered - inferential validity was achieved after 5-10 iterations, much earlier than indicated by the \(\widehat{R}\) and AC diagnostics. Of course, it never hurts to iterate longer, but such calculations hardly bring added value.{]}}

\bibliographystyle{sageh}
\bibliography{thesis.bib}


\end{document}
